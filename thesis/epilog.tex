\chapter*{Conclusion}
\addcontentsline{toc}{chapter}{Conclusion}
The main goal of this thesis was to try some existing or develop some problem-specific algorithms for the problem of feature detection in ionograms. The features to be detected were of linear type and we had to measure some of their properties (not only find them). The problem was complicated by relatively high levels of noise in ionograms and the fact that no features had to be present in a ionogram, or only some of them.

In our thesis we proposed a set of algorithms solving the feature detection problem. Some of the algorithms emerged as suitable for our detection problem while several others not so much. Unfortunately, none of the methods reaches the target error level of \n[\%]{1} which is believed to be the error level of the manually tagged data.

We believe further fine-tuning of parameters of the best algorithms would bring an even lower error level. Likely, investigation of more statistical properties of more ionograms could also give better insight into the initial filtering that discards ionograms not likely to contain any features. However, more manually tagged data would be needed for such test. Although more data exist, our university does not have access to them at the present. Processing of a considerably larger amount of data would also need a powerful computer or cluster of computers not available to us in the time of writing the thesis.

Further improvements that could help the algorithms a lot would be for example the ability to ``cancel out'' a detected feature. If we e.g. detect a ionospheric echo and have its trace, we should be able to determine the pixels belonging to the echo and setting them to the background value. However, this method has some caveats -- e.g. we must be pretty sure that the  pixels being cancelled out really belong to the feature and are not part of another one (or shared between more features). If we performed such cancelling, detection of the rest of features should be more precise. So we could advance from the features we are able to detect with the lowest error level to the ones with less reliable detection, improving the overall error for all features.

Another use of the algorithms that is not developed further in our thesis is combining them with the manually acquired data. Using the manual data as an initial guess for the algorithms, we could be able to get even more reliable results -- the problem is how to evaluate such improvement. Adding to that we do not think all the presented algorithms are able to utilize this initial guess, but some of them surely are. Moreover, we think there could exist better specific algorithms for such task. Incorporating such algorithms into the manual tagging tool could be a really helpful cooperation between men and computers. The human tagger would estimate the feature, the computer would then fine-tune it and the tagger could immediately check if the fit is better than his and thus discard erroneous detection results. 

In the course of writing our thesis we learned interesting information about Mars and space exploration. What is more important, we got a better and in-depth knowledge of some of the computer vision techniques learnt during university lectures. We find this enriching and helpful in our own further personal development. We would be really glad if our work found a real use in tagging further ionograms served by MARSIS.

Once again, we would like to thank many times to our supervisor RNDr.~Jana~�tanclov�,~Ph.D., who helped us with valuable comments and inspired us to use several methods not discovered by ourselves. Acknowledgements also belong the our consultant RNDr.~Franti�ek~N�mec,~Ph.D. who provided us with corrections of the physics-related part of work.