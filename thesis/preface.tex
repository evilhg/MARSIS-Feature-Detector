\chapter*{Introduction}
\addcontentsline{toc}{chapter}{Introduction}
Space exploration in its wider meaning has accompanied the mankind almost as long as e.g. writing. Looking at the stars, connecting them into constellations and thinking about the nature of the stars and the space between them -- all activities have been of interest for many people.

Up until the 50's of the \nth{20} century, these observations were limited to remote study. Telescopes (and very, very later spectroscopes, too) were the only ways of getting some useful information about cosmic bodies. Not a small number of information can be deduced from these observations, but sure there are many of them needing either better resolution unreachable from the Earth or requiring in situ measurements. To name some of the remotely observable properties, we can mention e.g. orbit measurements, mass estimates, gravity effect observations, overall chemical make-up determination (through spectroscopy) or deep-space radio waves analysis. With the development of technology many new procedures were engineered for measuring many other parameters of the cosmic bodies.

Starting with the successful launch of Sputnik~1, the first Earth's artificial satellite, the mankind entered the new epoch of in situ space exploration. Many spacecrafts have been launched since then with the most varied purposes and heading to all reachable interesting destinations in our Solar system (or staying at Earth's orbit). Some of them, like the Voyager~1 or Pioneer~10 are even heading out of the Solar system to explore the outer space.

After exploring the Moon (the NASA's Apollo missions) and Venus (USSR's Venera and Vega and NASA's Mariner, Magellan and Pioneer-Venus missions), the third nearest space neighbor of Earth was to be explored from close range -- Mars. Missions like Viking, Mars Global Surveyor, Mariner, Mars Odyssey or Fobos had Mars as their destination and target of observation.

In~2003 the European Space Agency~(ESA) launched its orbital planetary explorer called Mars~Express (MEX). Its primary target has been to detect traces of life on the planet, which became a very popular problem in the \nth{21} century. Although its primary mission ended in~2005, the probe has still (as of~2013) continued in operation and has been sending more and more scientific data to the Earth. Besides photos of the surface served in unrivaled resolution, it also proved presence of water and methane on Mars and in total has brought a lot of important scientific discoveries.

In our thesis we focus on one specific equipment of MEX -- the MARSIS radar. It is used for ionosphere and subsurface sounding using long radio waves. This is a technique allowing to detect subsurface structures as well as to determine the density profiles of the ionosphere. Using the subsurface sounding mode, MARSIS confirmed the existence of water ice under the surface. In particular, of our interest is the ionospheric sounding mode and the data it produces.

The ionospheric sounding data are received in the form of so called ``ionograms'' which can be imagined as two-dimensional images with intensity-coded data. In these images, several types of linear features are to be detected in order to extract the measured physical properties of ionosphere and plasma.

All these features have been extracted manually up to the present, which is on one hand very reliable, but on the other hand time-consuming and boring. The goal of our work is to examine several computer vision methods on this problem and try to find some performing the detection automatically as well as reliably. We aim at a conducting a comparative test of these methods and telling the pros and cons of each of them.

In Chapter~1 we describe the Mars Express orbiter in detail along with all its onboard experimental appliances. Chapter~2 defines the solved task precisely and analyses the ionograms and the general properties of the studied features. The subsequent chapters are then each dedicated to a different method of solving the task. We also present the outcomes of our experiments in these chapters. In the last chapter we provide a short summary of the presented techniques as well as their comparison and analysis of the obtained results. Finally, in the Conclusion part we conclude the thesis and its main results. We also discuss the possible directions of further work there.